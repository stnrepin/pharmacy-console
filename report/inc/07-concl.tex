\anonsection{Заключение}

В ходе выполнения курсового проекта были достигнуты все поставленные цели:
разработано техническое задание на ПК, проведено проектирование с применением
UML, создан ПК на ОО языке, написана программная документация различных видов.

Были получены ценнейшие знания о подходах и методах проектирования сложных
программных проектов, используя при этом современные и устоявшиеся технологии.
Так, в данной курсовой работе используются государственные стандарты,
методология разработки waterflow и различные виды UML-диаграмм. В совокупности
все это дает возможность создания приложений, полностью удовлетворяющих
потребности клиента, и делает процесс разработки абсолютно прозрачным как для
программиста, так и для клиента.

При непосредственной разработки программы было глубоко изучено практическое
применение объектно-ориентированного языка Java последних стандартов вместе с
фундаментальными фреймворками и библиотеками языка: JavaFX, Hibernate, JPA,
Jaspersoft, JUnit, Log4j и других. Знание и понимание этих технологий является
неотъемлемым требованием текущего рынка труда, предъявляемым Java-программистам.

Разработка документации также познакомила с существующими в этой области подходами.
Были изучены требования соответствующих государственных стандартов вкупе со
средствами, предоставляемыми экосистемой Java (например, Javadoc и
юнит-тестирование).


\clearpage

