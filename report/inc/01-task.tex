\begin{center}
    \uppercase{\textbf{Задание на курсовой проект}}
\end{center}

Студент Репин С.А. \\*
Группа 8307

\vspace{0.2cm}

Тема проекта: Разработка программного комплекса на языке Java

\vspace{0.2cm}

Исходные данные:\\*
Разработать ПК для администратора аптеки. В ПК должны храниться сведения о
болезнях и лекарствах. Администратор аптеки может добавлять,  изменять  и
удалять  эти  сведения.  Ему  может  потребоваться следующая информация:
\begin{itemize}
    \item какие лекарства применяются для лечения указанной болезни;
    \item имеется ли лекарство в аптеке и в каком количестве;
    \item какие лекарства и в каком количестве проданы за указанный период
            времени;
    \item на какую сумму проданы лекарства за месяц.
\end{itemize}

\vspace{0.2cm}

Содержание пояснительной записки:\\*
<<Техническое задание>>, <<Описание процесса проектирования ПК>>, <<Руководство
оператора>>, <<Исходные тексты ПК>>, <<Заключение>>, <<Список использованных
источников>>.

\vspace{0.2cm}

Предполагаемый объем пояснительной записки: \\*
Не менее 20 страниц.

\vspace{0.2cm}

\begin{tabular}{@{}ll}
    Дата выдачи задания: & 21.09.2020 \\
    Дата сдачи курсового проекта: & 27.12.2020 \\
    Дата защиты курсового проекта: & \phantom{21}.12.2020 \\
\end{tabular}

\vspace{0.2cm}

\begin{tabularx}{\textwidth}{@{}lcXr}
    Студент гр. 8307 & \hspace{1.6cm} & \rule{5cm}{1pt} & Репин С.А.
\end{tabularx}

\begin{tabularx}{\textwidth}{@{}lcXr}
    Преподаватель & \hspace{2cm} & \rule{5cm}{1pt} & Гречухин М.Н.
\end{tabularx}



\clearpage
