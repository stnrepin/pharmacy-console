\section{Руководство оператора}


\subsection{Назначение программы}

Программный комплекс Pharmacy Console предназначен для повышения эффективности
и устойчивости работы аптеки с помощью создания единого интерфейса для
выполнения основных задач, возникающих перед администратором аптеки.

\subsection{Условия выполнения программы}

Минимальный состав аппаратных средств:
\begin{itemize}
    \item процессор архитектуры x86-64 с тактовой частотой 2.5 ГГц;
    \item оперативная память объемом 2 Гб;
    \item жесткий диск объемом 64 Гб;
    \item графический адаптер;
    \item монитор с разрешением экрана 1024x768 пикселей.
\end{itemize}

Необходимый набор программных средств:
\begin{itemize}
    \item ОС Windows (7, 8.1, 10) или GNU/Linux (ядро старше 3.2);
    \item JDK 14;
    \item PostgreSQL 12 сервер.
\end{itemize}

Конечный пользователь программы (оператор) должен обладать
практическими навыками работы с графическим пользовательским интерфейсом
операционной системы, должен быть аттестован минимум на II квалификационную
группу по электробезопасности, иметь квалификацию <<Пользователь ЭВМ>>, владеть
английским языков на уровне чтения простых фраз.

\subsection{Описание задачи}

Программный комплекс хранит сведения о болезнях и лекарствах. Оператор может
добавлять, изменять и удалять эти сведения. Ему доступны отдельно следующие
функции:
\begin{itemize}
    \item получение списка лекарств для лечения указанной болезни;
    \item проверка наличия лекарства в аптеке;
    \item получения количества доступный единиц лекарства;
    \item получение информации о количестве и общей сумме проданных лекарств за
        указанный период времени.
\end{itemize}

Для решения задачи используется разработанное приложение, предоставляющее
возможность выполнять указанные выше операции над сведениями о болезнях,
лекарствах и заказах лекарств, которые располагаются в базе данных, через
взаимодействие с графическим интерфейсом.

\subsection{Входные и выходные данные}

Входными данными программы является набор действий оператора с элементами
графического интерфейса (нажатия на кнопки, ввод текста в текстовые поля,
включение/выключение флагов).

Выходные данные --- изменения базы данных приложения, с последующим обновлением
отображаемой интерфейсом информации (добавление или удаление строк таблицы,
показ диалоговых окон). При печати отчета выходными данными также выступает
результирующий PDF-файл отчета.

\subsection{Выполнение программы}

\subsubsection{Запуск}

Запуск программы осуществляется с помощью выполнения JAR-файл. Перед его
запуском необходимо удостовериться в том, что сервер БД запущен.

При первом запуске программы появится изображенное ниже окно.

\addimghere{res/docs/new-win.png}{1}{Пример главного окна с открытой вкладной
    <<Лекарства>>}{}

\subsubsection{Выполнение программы}

Рассмотрим основные функции программы и их исполнение с помощью элементов
управления.
\begin{itemize}
    \item Переход к списку лекарств

        Чтобы перейти к списку лекарств (и в дальнейшем работать с ним)
        необходимо нажать на кнопку <<Medicine>>, что переключит текущую
        вкладку на вкладку <<Лекарства>>.

    \item Добавление нового лекарства

        Нажмите кнопку с изображением <<+>>, откроется окно
        <<Добавление/редактирование лекарства>>, введите в него необходимые
        сведения (имя не должно быть пустой строкой, болезни можно перечислять
        через <<;>>) и нажмите кнопку <<OK>>. Для отмены нажмите кнопку
        <<Cancel>>. Новое лекарство будет добавлено в таблицу.

    \item Удаление лекарства

        Выделите в таблице строчку, соответствующую необходимому лекарству.
        Нажмите кнопку с изображением <<->>. Подтвердите свое действие.
        Лекарство исчезнет из таблицы.

    \item Редактирование лекарства

        Выделите в таблице строчку, соответствующую необходимому лекарству.
        Нажмите кнопку с изображением карандаша, откроется окно
        <<Добавление/редактирование лекарства>>, измените в нем необходимые
        сведения (на них накладываются такие же требования, как и в функции
        <<Добавление нового лекарства>>) и нажмите кнопку <<OK>>. Для отмены
        нажмите кнопку <<Cancel>>. Сведения о лекарстве будут обновлены в
        таблице.

    \item Создание заказа

        Выделите в таблице строчку, соответствующую необходимому лекарству.
        Нажмите кнопку с изображением тачки и знаком <<+>>. Откроется окно
        <<Добавление заказа>>, введите в нем необходимое число единиц
        заказываемого лекарства. Нажмите кнопку <<OK>>. Для отмены нажмите
        кнопку <<Cancel>>. В результате количество доступных единиц лекарства в
        таблице лекарств уменьшится, а во вкладке <<Заказы>> в таблице заказов
        будет добавлен новый заказ, информация об общей сумме заказов
        обновится.

    \item Выборка лекарств для лечения указанной болезни

        Введите название интересующей болезни в текстовое поле рядом с флагом
        (переключателем) на панели "Фильтр". Активируйте переключатель. В
        таблице останутся только те лекарства, которые применяются для лечения
        указанной болезни. Чтобы вернуть общий список лекарств, деактивируйте
        переключатель.

    \item Переход к списку болезней

        Чтобы перейти к списку болезней (и в дальнейшем работать с ним)
        необходимо нажать на кнопку <<Disease>>, что переключит текущую вкладку
        на вкладку <<Болезни>>.

    \item Добавление новой болезни

        Нажмите кнопку с изображением <<+>>, откроется окно
        <<Добавление/редактирование болезни>>, введите в него необходимые
        сведения (имя не должно быть пустой строкой и не должно содержать
        символ <<;>>) и нажмите кнопку <<OK>>. Для отмены нажмите кнопку
        <<Cancel>>. Новая болезнь будет добавлена в таблицу.

    \item Удаление болезни

        Выделите в таблице строчку, соответствующую необходимой болезни.
        Нажмите кнопку с изображением <<->>. Подтвердите свое действие.
        Болезнь исчезнет из таблицы. Указанное действие можно выполнить только
        если нет лекарств, ссылающихся на данную болезнь.

    \item Редактирование болезни

        Выделите в таблице строчку, соответствующею необходимой болезни.
        Нажмите кнопку с изображением карандаша, откроется окно
        <<Добавление/редактирование болезни>>, измените в нем необходимые
        сведения (на них накладываются такие же требования, как и в функции
        <<Добавление новой болезни>>) и нажмите кнопку <<OK>>. Для отмены
        нажмите кнопку <<Cancel>>. Сведения о болезни будут обновлены в
        таблице.

    \item Переход к списку заказов

        Чтобы перейти к списку болезней (и в дальнейшем работать с ним)
        необходимо нажать на кнопку <<Medicine Order>>, что переключит текущую
        вкладку на вкладку <<Заказы>>.

    \item Выборка заказов по дате

        На панели <<Фильтр>> выберете поочередно даты начала периода и его
        конца (нажмите на кнопки с изображением календаря рядом с полем для
        ввода, откроется всплывающее окно выбора даты). Активируйте
        переключатель. Список заказов в таблице обновится, как и общая
        сумма заказов.

    \item Печать отчета о заказах

        Нажмите кнопку с изображением принтера (она единственная на форме).
        Откроется диалог выбора файл для сохранения. Выберете нужный файл и
        нажмите <<OK>> (или <<Cancel>>, чтобы отменить печать). Отобразится
        элемент управления Спиннер, показывающий факт выполнения операции (она
        может продлиться некоторое время). В выбранном файле появится отчет.
\end{itemize}

Скриншоты выполнения некоторых функций представлены ниже. Работа с остальными
функциями происходит аналогично.

\addimghere{res/docs/add-win.png}{1}{Добавление/редактирование лекарства}{}
\addimghere{res/docs/add-order-win.png}{1}{Добавление нового заказа}{}
\addimghere{res/docs/filtered.png}{1}{Фильтрация лекарств по болезни}{}
\addimghere{res/docs/rem-win.png}{1}
{Окно ожидания подтверждения об удалении лекарства}{}

\subsubsection{Завершение программы}

Для завершения программы необходимо закрыть приложение штатными средствами ОС.

\subsection{Проверка программы}

В целях проверки программы следует выполнить следующие контрольные примеры:
\begin{enumerate}
    \item Добавить лекарство c именем <<M>>, ссылающееся на болезни <<D1>> и
        <<D2>> разделе  «Входные  и  выходные  данные»  должны  быть  указаны
        сведения о входных и выходных д

        \textit{Ожидаемое поведение:} В таблице с лекарствами появится новое
        лекарство, а в таблице с болезнями появятся две новые болезни.

    \item Выполнить действия аналогичные 1, но сперва добавить болезни
        (лекарство <<M2>>, болезни <<D3>> и <<D4>>)

        \textit{Ожидаемое поведение:} Аналогично 1, но новые лекарства не
        создаются.

    \item Отредактировать численные поля существующего лекарства, изменить
        название болезни

        \textit{Ожидаемое поведение:} Численные поля изменяться в таблице,
        добавится новая болезнь, старая останется без изменений.

    \item Произвести фильтрацию лекарств по имени существующей и несуществующей
        болезни. Отменить фильтрацию

        \textit{Ожидаемое поведение:} При фильтрации по существующей болезни
        отобразятся только те лекарства, которые ссылаются на указанную
        болезнь. При фильтрации по несуществующей болезни таблица окажется
        пустой. При отмене фильтрации таблица вернется в исходное состояние.

    \item Отредактировать название болезни, на которую ссылается лекарство

        \textit{Ожидаемое поведение:} В таблице с болезнями и в свойствах
        лекарства изменится имя данной болезни.

    \item Создать заказ

        \textit{Ожидаемое поведение:} В таблице с заказами появится заказ с
        указанными параметрами (проверить соответствие названия лекарства, цены
        за штуку и общего количества единиц в заказе). В таблице лекарств для
        заказанного лекарства будет уменьшено значение поля доступного
        количества единиц.

    \item Произвести фильтрацию заказов

        \textit{Ожидаемое поведение:} в таблицу заказов отобразятся только
        заказы, совершенные в указанный период (смотреть по столбцу <<Order
        Date>>).

    \item Удалить лекарство, для которой был создан заказ

        \textit{Ожидаемое поведение:} В поле <<Medicine>> заказа, ссылающегося
        на удаленное лекарство, таблицы заказов появится надпись <<N/A>> вместо
        имени лекарства.

    \item Отфильтровать заказы и создать отчет

        \textit{Ожидаемое поведение:} в отчете окажутся только заказы за
        указанный период.

\end{enumerate}

Каждый контрольный пример следует выполнять после перезагрузки программы, а
ожидаемое поведение сверять до перезагрузки и сразу же после.


\subsection{Сообщения оператору}

В соответствии с идеологией Unix-way программа не выводит сообщения при удачном
совершении действия (предполагается, что оператор понимает, что он делает). В
иных случаях, а также в случаях, когда действия оператора меняют содержимое
формы, отображаются такие сообщения:
\begin{itemize}
    \item Окно <<Ошибка подключения к БД>>

        Возникает, если при запуске программы не удается установить сообщение с
        сервером БД.

        Вероятнее всего, не запущен сервер базы данных. Необходимо запустить
        сервер и затем заново запустить программу.

    \item Окно <<Возникла ошибка>>

        Отображается диалоговое окно ошибки, с описанием ошибки (также, если
        возможно приводится стек вызовов).

        Если программа остается работоспособной (не завершается после нажатия
        на <<OK>>), то от оператора требуется только повторить при
        необходимости выполненные последний раз действия с программой. В
        ином случае, потребуется повторный запуск программы.

    \item Окно <<Подтверждение удаления>>

        Перед удалением лекарства или болезни у пользователя потребуют
        подтверждение указанного действия (нажать <<OK>>, для совершения
        удаления, иначе -- <<Cancel>>).

    \item Окно <<Некорректные вводимые данные>>

        При добавлении или редактировании лекарства (болезни) производится
        валидация вводимых данных. В случае нарушения соглашения о формате
        вводимых данных оператору будет отображено соответствующее сообщение.
        От него требуется перепроверить введенные данные.
\end{itemize}

Примеры сообщений приведены на рисунках ниже.

\addimghere{res/docs/add-err.png}{1}
{Сообщение об ошибке при попытки добавления лекарства с пустым именем}{}
\addimghere{res/docs/add-err-d.png}{1}
{Сообщение об ошибке при попытки добавления болезни с пустым именем}{}
\addimghere{res/docs/db-err.png}{1}
{Сообщение об ошибке при неудачной попытки соединения с БД}{}


\clearpage

