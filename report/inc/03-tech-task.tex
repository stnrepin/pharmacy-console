\section{Техническое задание}



\subsection{Введение}


Наименование программы --- <<Pharmacy Console>>.

Программный комплекс предназначен для цифровизации работы аптеки. Он облегчает
администратору (кассиру-консультанту) аптеки выполнение своих обязанностей,
повышая его эффективность.


\subsection{Основания для разработки}


Основанием для разработки является выполнение курсового проектирования как
результата прохождения курса <<Объектно-ориентированное программирование>>.
Тема работы --- <<Разработка ПК для администратора аптеки>>, шифр темы --- 16.


\subsection{Назначение разработки}


Программа предоставляет возможности для учета товаров аптеки, ассистирования
при выборе лекарств для покупки, ведения бухгалтерского учета заказов. Основным
пользователем выступает администратор (кассир-консультант) аптеки, эксплуатация
программы предполагает использование штатных ЭВМ организации.


\subsection{Требования к программе}


\subsubsection{Требования к функциональным характеристикам}

Программа представляет собой графическое приложение, хранящее сведения о
болезнях и лекарствах. Пользователь программы, взаимодействуя элементами
управления, может добавлять, изменять и удалять эти сведения.

Кроме того, пользователь имеет возможность легко получать следующую
информацию:
\begin{itemize}
    \item какие лекарства применяются для лечения указанного заболевания;
    \item имеется ли лекарство в аптеке и в каком количестве;
    \item какие лекарства и в каком количестве проданы за указанный период;
    \item на какую сумму проданы лекарства за месяц.
\end{itemize}

Входными данными выступают данные, вводимые пользователем с клавиатуры, а также
с помощью элементов управления. Выходные данные --- таблицы, списки,
всплывающие окна и  изменения базы данных приложения.

Данные должны надежно храниться в СУБД. Должна производиться валидация входных
данных (имена должны быть непустыми, количественные характеристики должны быть
натуральными числами и т.д.).

Программа должна работать безотказно и не допускать падений.

С учетом развития электронных устройств особых требований ко времени отклика
программы не ставится, она лишь не должна иметь заметных для обычного
пользователя лагов интерфейса. Время запуска приложения --- не более 5 секунд.

\subsubsection{Требования к надежности}

Надежное (устойчивое) функционирование программы обеспечивается выполнением
пользователем следующего набора мероприятий:
\begin{itemize}
    \item организацией бесперебойного питания технических средств;
    \item соблюдением условий эксплуатации;
    \item выполнением требований и рекомендаций государственных актов в области
        использования и обслуживания программных комплексов;
    \item своевременным обновлением программного (в том числе, данной
        программы) и аппаратного обеспечения;
\end{itemize}

\subsubsection{Условия эксплуатации}

Программа запускается на компьютерах аптеки для каждого администратора. Перед
запуском программы необходимо запустить сервер СУБД. Должна обеспечиваться
надежная и бесперебойная работа сервера.

Непосредственные пользователи программы должны владеть основами компьютерной
грамотности и знать английский язык на уровне чтения. Для установки, первичной
настройки и сервисного обслуживания может потребоваться системный
администратор, обладающий навыками установки программ в текущую операционную
систему и развертывания СУБД.

\subsubsection{Требования к составу и параметрам технических средств}

Минимальный состав аппаратных средств:
\begin{itemize}
    \item процессор архитектуры x86-64 с тактовой частотой, 2.5 ГГц;
    \item оперативная память объемом, 2 Гб;
    \item жесткий диск объемом 64 Гб;
    \item графический адаптер;
    \item монитор с разрешением экрана 1024 x 768 пикселей.
\end{itemize}

\subsubsection{Требования к информационной и программной совместимости}

\begin{itemize}
    \item Программа работает под операционными системами Windows и GNU/Linux,
        удовлетворяющими требованиям JDK 14.
    \item Требуются установленные программы: JDK 14 и PostgreSQL 12.
\end{itemize}

Исходный код должен быть написан на языке Java 14, собираться средствами Maven,
вести журналирование своей работы.

\subsection{Требования к программной документации}

Предварительный состав программной документации:
\begin{itemize}
    \item техническое задание;
    \item документация к исходному коду в формате HTML;
    \item руководство оператора;
    \item пояснительная записка.
\end{itemize}


\subsection{Стадии и этапы разработки}


Выделяются следующие стадии и их этапы:
\begin{enumerate}
    \item техническое задание:
        \begin{itemize}
            \item разработка ТЗ;
            \item согласование ТЗ;
            \item утверждение ТЗ;
        \end{itemize}
    \item технический проект:
        \begin{itemize}
            \item разработка программы;
            \item разработка документации;
            \item тестирование программы и документации;
        \end{itemize}
    \item внедрение:
        \begin{itemize}
            \item подготовка к передаче;
            \item передача программы.
        \end{itemize}
\end{enumerate}

На этапе разработки технического задания можно выделить такие работы:
\begin{itemize}
    \item постановка задачи;
    \item определение и уточнение требований к техническим средствам;
    \item определение требований к программе;
    \item определение стадий, этапов и сроков разработки программы и
        документации на нее;
    \item согласование и утверждение технического задания.
\end{itemize}

На этапа разработки программного продукта должны быть выполнены работы по
написанию кода программы, модульных тестов к ней и ее отладка.

На этапе испытаний программы должны быть выполнены такие работы, как:
\begin{itemize}
    \item разработка, согласование и утверждение порядка и методики испытаний;
    \item проведение приемо-сдаточных испытаний;
    \item корректировка программы и программной документации по результатам
        испытаний.
\end{itemize}

Срок выполнения работы: 21.09.2020-29.12.2020.

Исполнитель: Репин Степан Александрович.

\subsection{Порядок контроля и приемки}

Должно быть проведено ручное тестирование программного комплекса при различным
условиях на соответствие настоящему техническому заданию, проверка исходного
кода на соответствие документации и применения модульных тестов как
спецификации исходного кода.



\clearpage

