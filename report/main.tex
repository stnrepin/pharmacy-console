\documentclass[a4paper,14pt]{extarticle}
\input{/home/acesk/Documents/latex/preamble.tex}

\usepackage{fancyvrb}

\newcommand{\Code}[1]{\textit{#1}}

\setlength{\extrarowheight}{.5ex}

\begin{document}

\begin{titlepage}
\begin{center}
    \uppercase{\textbf{Минобрнауки России\\
            Санкт-Петербургский государственный\\
            электротехнический университет\\
            «ЛЭТИ» им. В.И.Ульянова (Ленина)
    }}
    \vspace{0.25cm}

    \textbf{Кафедра ВТ}
    \vfill

    \uppercase{\textbf{\large{
        Курсовой проект
    }}}
    \\
    \textbf{\large{
      по дисциплине «Объектно-ориентированное программирование»\\
      Тема: Разработка программного комплекса на языке Java\\
      \vspace{0.5cm}
    }}
  \bigskip
\end{center}
\vfill

\begin{tabularx}{\textwidth}{@{}lcXr}
    Студент гр. 8307 & \hspace{1.6cm} & \rule{5cm}{1pt} & Репин С.А.
\end{tabularx}

\vspace{0.5cm}

\noindent
\begin{tabularx}{\textwidth}{@{}lcXr}
    Преподаватель & \hspace{2cm} & \rule{5cm}{1pt} & Гречухин М.Н.
\end{tabularx}

\hfill \break
\hfill \break

\begin{center}
  Санкт-Петербург\\2020
\end{center}

\end{titlepage}



\renewcommand*{\thepage}{}
\tableofcontents
\clearpage
\renewcommand*{\thepage}{\arabic{page}}

\setcounter{page}{3}

\anonsection{Цель работы}

Реализовать слой взаимодействия с базой данных, в соответствии со схемой,
разработанной в предыдущей лабораторной работе, для решения следующей задачи.

\anonsection{Задание}

Разработать ПК для администратора аптеки. В ПК должны
храниться сведения о болезнях и лекарствах. Администратор аптеки может
добавлять, изменять и удалять эти сведения. Ему может потребоваться
следующая информация:
\begin{itemize}
    \item какие лекарства применяются для лечения указанной болезни;
    \item имеется ли лекарство в аптеке и в каком количестве;
    \item какие лекарства и в каком количестве проданы за указанный период
        времени;
    \item на какую сумму проданы лекарства за месяц.
\end{itemize}

\anonsection{Введение}

При выполнении лабораторной работы был реализован функционал пустых классов,
ставших результатом лабораторной работы №1. Исходный код проекта доступен на GitHub
\footnote{\url{https://github.com/stnrepin/oop_labs/tree/lab2/}}.

Проект предназначен для использования в IDE IntelliJ IDEA, собирается с помощью
Maven. Используется Java 14 и PostgreSQL 12.

\clearpage


\section{Пояснения к выполнению работы}

В основном был реализован функционал (методы) DAO-классов, которые выполняют
различные запросы к базе данных, на основе структуры и данных классов-моделей.
На этот раз были добавлены интерфейсы для всех DAO-классов.  Также, был
реализован класс-сервис, объединяющий в себе все DAO-классы, выступая в
фасадом к ним.

С целью повышения удобства использования создание сессии для базы данных было
вынесено в отдельный вспомогательный класс \Code{PersistenceEntityManagerUtils},
который предоставляет интерфейс к текущей сессии DAO-классом через
callback-и.

Для реализации сложных запрос, отличающихся от стандартных CRUD, был
использован JPQL -- особый подвид языка SQL, позволяющий оперировать объектами
Java внутри привычного SQL.


\clearpage


\anonsection{Вывод}

В ходе выполнения данной лабораторной работы получены знания и практические
навыки в работе с базой данных в Java с помощью Java Persistence API, используя
объектный ORM-подход. Были получены навыки использования JPQL как
высокоуровневого аналога SQL.

\end{document}

