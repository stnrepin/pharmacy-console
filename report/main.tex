\documentclass[a4paper,14pt]{extarticle}
\input{/home/acesk/Documents/latex/preamble.tex}

\usepackage{fancyvrb}

\newcommand{\Code}[1]{\textit{#1}}

\setlength{\extrarowheight}{.5ex}

\begin{document}

\begin{titlepage}
\begin{center}
    \uppercase{\textbf{Минобрнауки России\\
            Санкт-Петербургский государственный\\
            электротехнический университет\\
            «ЛЭТИ» им. В.И.Ульянова (Ленина)
    }}
    \vspace{0.25cm}

    \textbf{Кафедра ВТ}
    \vfill

    \uppercase{\textbf{\large{
        Курсовой проект
    }}}
    \\
    \textbf{\large{
      по дисциплине «Объектно-ориентированное программирование»\\
      Тема: Разработка программного комплекса на языке Java\\
      \vspace{0.5cm}
    }}
  \bigskip
\end{center}
\vfill

\begin{tabularx}{\textwidth}{@{}lcXr}
    Студент гр. 8307 & \hspace{1.6cm} & \rule{5cm}{1pt} & Репин С.А.
\end{tabularx}

\vspace{0.5cm}

\noindent
\begin{tabularx}{\textwidth}{@{}lcXr}
    Преподаватель & \hspace{2cm} & \rule{5cm}{1pt} & Гречухин М.Н.
\end{tabularx}

\hfill \break
\hfill \break

\begin{center}
  Санкт-Петербург\\2020
\end{center}

\end{titlepage}



\renewcommand*{\thepage}{}
\tableofcontents
\clearpage
\renewcommand*{\thepage}{\arabic{page}}

\setcounter{page}{3}

\anonsection{Цель работы}

Реализация фронтенд-части приложения. Знакомство с правилами построения
экранных форм.

\anonsection{Задание}

Разработать графических интерфейс для приложения, соответствующего следующим
требованиям.

Разработать ПК для администратора аптеки. В ПК должны
храниться сведения о болезнях и лекарствах. Администратор аптеки может
добавлять, изменять и удалять эти сведения. Ему может потребоваться
следующая информация:
\begin{itemize}
    \item какие лекарства применяются для лечения указанной болезни;
    \item имеется ли лекарство в аптеке и в каком количестве;
    \item какие лекарства и в каком количестве проданы за указанный период
        времени;
    \item на какую сумму проданы лекарства за месяц.
\end{itemize}

\anonsection{Введение}

При выполнении лабораторной работы был созданы файлы формата FXML и CSS (LESS),
описывающие интерфейс программы (соответствуют View в паттерне MVC). Исходный
код проекта доступен на GitHub
\footnote{\url{https://github.com/stnrepin/oop_labs/tree/lab3/}}.

Проект предназначен для использования в IDE IntelliJ IDEA, собирается с помощью
Maven. Используется Java 14, PostgreSQL 12 и JavaFX 14.

\clearpage


\section{Пояснения к выполнению работы}

\subsection{Прототипировние UI}

В качестве сервиса для создания прототипа была выбрана
Figma\footnote{\url{https://www.figma.com/}} по критерию современности,
популярности и гибкости. Графический интерфейс был разработан с применением
стиля Material Design. Результат представлен на рисунке ниже.

\addimghere{res/ui_proto.png}{1}{}{}


\clearpage

\subsection{Реализация дизайна в Java}

В качестве GUI-фреймворка был выбран JavaFX с библиотекой JFoenix, позволяющей
применять стили Material Design.

Для создания гибкой и расширяемой архитектуры применяется шаблон проектирования
MVC, который дает возможность отделить интерфейс приложения от его
бизнес-логики, что в свою очередь предоставляет легкость тестирования и
добавления новых форм и окон. В данной лабораторной работе была разработана
часть Представления (View), реализованная в специальных файлах FXML-разметки,
специфичных для JavaFX, и файлах CSS-стилей (учитывая различные неудобства
использования чистого CSS, было принято решения писать стили в LESS, который
затем компилируется в CSS с помощью созданной для этого цели Maven).

Исходя из предметной области, получилось четыре представления: \Code{MainView},
\Code{MedicineView}, \Code{DiseaseView}, \Code{MedicineOrderView}, из которых
в данной лабораторной работе были реализованы первые два.

\addimghere{res/ui_small.png}{1}{Результат создания UI в Java}{}

\anonsection{Вывод}

В ходе выполнения данной лабораторной работы получены знания и практические
навыки в разработке, прототипировании и создании графических интерфейсов в Java
с использованием современных для этого средств. Были получены умения в
проектировании архитектуры GUI-приложений с применением MVC, а также работе с
системой сборки Maven.

\end{document}

