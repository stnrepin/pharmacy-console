\documentclass[a4paper,14pt]{extarticle}
\input{/home/acesk/Documents/latex/preamble.tex}

\usepackage{fancyvrb}

\newcommand{\Code}[1]{\textit{#1}}

\setlength{\extrarowheight}{.5ex}

\begin{document}

\begin{titlepage}
\begin{center}
    \uppercase{\textbf{Минобрнауки России\\
            Санкт-Петербургский государственный\\
            электротехнический университет\\
            «ЛЭТИ» им. В.И.Ульянова (Ленина)
    }}
    \vspace{0.25cm}

    \textbf{Кафедра ВТ}
    \vfill

    \uppercase{\textbf{\large{
        Курсовой проект
    }}}
    \\
    \textbf{\large{
      по дисциплине «Объектно-ориентированное программирование»\\
      Тема: Разработка программного комплекса на языке Java\\
      \vspace{0.5cm}
    }}
  \bigskip
\end{center}
\vfill

\begin{tabularx}{\textwidth}{@{}lcXr}
    Студент гр. 8307 & \hspace{1.6cm} & \rule{5cm}{1pt} & Репин С.А.
\end{tabularx}

\vspace{0.5cm}

\noindent
\begin{tabularx}{\textwidth}{@{}lcXr}
    Преподаватель & \hspace{2cm} & \rule{5cm}{1pt} & Гречухин М.Н.
\end{tabularx}

\hfill \break
\hfill \break

\begin{center}
  Санкт-Петербург\\2020
\end{center}

\end{titlepage}



\renewcommand*{\thepage}{}
\tableofcontents
\clearpage
\renewcommand*{\thepage}{\arabic{page}}

\setcounter{page}{3}

\anonsection{Цель работы}

Разработать систему классов (и изобразить ее на UML-диаграмме) для решения
следующей задачи.

\anonsection{Задание}

Разработать ПК для администратора аптеки. В ПК должны
храниться сведения о болезнях и лекарствах. Администратор аптеки может
добавлять, изменять и удалять эти сведения. Ему может потребоваться
следующая информация:
\begin{itemize}
    \item какие лекарства применяются для лечения указанной болезни;
    \item имеется ли лекарство в аптеке и в каком количестве;
    \item какие лекарства и в каком количестве проданы за указанный период
        времени;
    \item на какую сумму проданы лекарства за месяц.
\end{itemize}

\anonsection{Введение}

При выполнении лабораторной работы была создана UML-диаграмма классов,
находящихся в слоях доступа к базе данных и бизнес-логики.
На основе этой диаграммы классов был создан набор пустых (stub) классов.
Исходный код проекта доступен на GitHub
\footnote{\url{https://github.com/stnrepin/oop_labs/tree/master/}}.

Проект предназначен для использования в IDE IntelliJ IDEA, собирается с помощью
Maven. Используется Java 14.

\clearpage


\section{Пояснения к выполнению работы}

В начале были определены сущности предметной области и их свойства: Лекарство
(\Code{Medicine}), Заказ лекарства (класс \Code{MedicineOrder}), Болезнь (класс
\Code{Disease}). Для этих сущностей, затем, были созданы Data access objects
(DAO) классы, предназначенные для выполнения операций над сущностями через
запросы к базе данных. Также, был добавлен специальный класс
\Code{PharmacyServiceImpl}, который, используя DAO-классы, реализует
непосредственно бизнес-логику приложения (конкретно пункты требований из
Задания).

Заметим, что у всех DAO-классов есть общие операции, обозначаемые аббревиатурой
CRUD (от create, read, update, delete). Реализация этих операций для всех DAO
будет одинаковой, поэтому соответствующие методы были выделены в отдельный
обобщенный класс \Code{CrudDaoOperations<K, E>}.

Все классы были разделены на пакеты: \Code{models} для моделей,
\Code{dao.hibernate} для DAO-классов и \Code{services} для сервисов.

Стоит сказать, в соответствии с принципом инверсии зависимостей есть смысл
ввести интерфейсы DAO- и сервис-классов (тем более, оно пригодится для
мокирования при unit-тестировании), но в данный момент это не нужно и только
лишь усложнит диграммы, не принося пользы. Поэтому было решено добавить
интерфейсы при необходимости.

\addimghere{res/uml-diagram.png}{1}{Диаграмма классов}{}

\clearpage


\anonsection{Вывод}

В ходе выполнения данной лабораторной работы получены знания и практические
навыки в разработке архитектуры приложений для работы с базой данных через
ORM-подход. Изучены общепринятые подходы дизайна слоя доступа к
базе данных в языке Java. Получены знания в работе с UML-диаграммами классов.

\end{document}

